\documentclass[]{book}
\usepackage{lmodern}
\usepackage{amssymb,amsmath}
\usepackage{ifxetex,ifluatex}
\usepackage{fixltx2e} % provides \textsubscript
\ifnum 0\ifxetex 1\fi\ifluatex 1\fi=0 % if pdftex
  \usepackage[T1]{fontenc}
  \usepackage[utf8]{inputenc}
\else % if luatex or xelatex
  \ifxetex
    \usepackage{mathspec}
  \else
    \usepackage{fontspec}
  \fi
  \defaultfontfeatures{Ligatures=TeX,Scale=MatchLowercase}
\fi
% use upquote if available, for straight quotes in verbatim environments
\IfFileExists{upquote.sty}{\usepackage{upquote}}{}
% use microtype if available
\IfFileExists{microtype.sty}{%
\usepackage{microtype}
\UseMicrotypeSet[protrusion]{basicmath} % disable protrusion for tt fonts
}{}
\usepackage[margin=1in]{geometry}
\usepackage{hyperref}
\hypersetup{unicode=true,
            pdftitle={Principles for modelling packages},
            pdfauthor={TBD},
            pdfborder={0 0 0},
            breaklinks=true}
\urlstyle{same}  % don't use monospace font for urls
\usepackage{natbib}
\bibliographystyle{apalike}
\usepackage{longtable,booktabs}
\usepackage{graphicx,grffile}
\makeatletter
\def\maxwidth{\ifdim\Gin@nat@width>\linewidth\linewidth\else\Gin@nat@width\fi}
\def\maxheight{\ifdim\Gin@nat@height>\textheight\textheight\else\Gin@nat@height\fi}
\makeatother
% Scale images if necessary, so that they will not overflow the page
% margins by default, and it is still possible to overwrite the defaults
% using explicit options in \includegraphics[width, height, ...]{}
\setkeys{Gin}{width=\maxwidth,height=\maxheight,keepaspectratio}
\IfFileExists{parskip.sty}{%
\usepackage{parskip}
}{% else
\setlength{\parindent}{0pt}
\setlength{\parskip}{6pt plus 2pt minus 1pt}
}
\setlength{\emergencystretch}{3em}  % prevent overfull lines
\providecommand{\tightlist}{%
  \setlength{\itemsep}{0pt}\setlength{\parskip}{0pt}}
\setcounter{secnumdepth}{5}
% Redefines (sub)paragraphs to behave more like sections
\ifx\paragraph\undefined\else
\let\oldparagraph\paragraph
\renewcommand{\paragraph}[1]{\oldparagraph{#1}\mbox{}}
\fi
\ifx\subparagraph\undefined\else
\let\oldsubparagraph\subparagraph
\renewcommand{\subparagraph}[1]{\oldsubparagraph{#1}\mbox{}}
\fi

%%% Use protect on footnotes to avoid problems with footnotes in titles
\let\rmarkdownfootnote\footnote%
\def\footnote{\protect\rmarkdownfootnote}

%%% Change title format to be more compact
\usepackage{titling}

% Create subtitle command for use in maketitle
\newcommand{\subtitle}[1]{
  \posttitle{
    \begin{center}\large#1\end{center}
    }
}

\setlength{\droptitle}{-2em}

  \title{Principles for modelling packages}
    \pretitle{\vspace{\droptitle}\centering\huge}
  \posttitle{\par}
    \author{TBD}
    \preauthor{\centering\large\emph}
  \postauthor{\par}
      \predate{\centering\large\emph}
  \postdate{\par}
    \date{2018-07-06}


\usepackage{amsthm}
\newtheorem{theorem}{Theorem}[chapter]
\newtheorem{lemma}{Lemma}[chapter]
\theoremstyle{definition}
\newtheorem{definition}{Definition}[chapter]
\newtheorem{corollary}{Corollary}[chapter]
\newtheorem{proposition}{Proposition}[chapter]
\theoremstyle{definition}
\newtheorem{example}{Example}[chapter]
\theoremstyle{definition}
\newtheorem{exercise}{Exercise}[chapter]
\theoremstyle{remark}
\newtheorem*{remark}{Remark}
\newtheorem*{solution}{Solution}
\begin{document}
\maketitle

{
\setcounter{tocdepth}{1}
\tableofcontents
}
\chapter{Intro}\label{intro}

\textbf{Rule 1}: Always spell it \emph{modelling}, never
\emph{modeling}.

\chapter{Conceptual overview of
modelling}\label{conceptual-overview-of-modelling}

\begin{itemize}
\item
  what is a model: models, estimands, estimators and model
  specifications
\item
  what do we do with models
\item
  how do fit models
\item
  once we have a fit model, how do we predict or do inference
\item
  the difference between working with a single fit vs a set of fits.
  LASSO example: wanting to use the coefficients for prediction vs
  wanting to see the order in which features enter the model
\end{itemize}

\chapter{Danger Zone}\label{danger-zone}

little things to include somewhere: - the danger of misspecified
arguments disappearing into \texttt{...}

\chapter{Data Specification}\label{data-specification}

\begin{itemize}
\item
  formulas, model.frame, term objects, etc
\item
  data / design matrix specification - \texttt{recipes}
\end{itemize}

habit: get the df right, then y \textasciitilde{} . in the formula.
would be nice to still see the features in the call?

\begin{itemize}
\tightlist
\item
  ask users to use data.frames and tibbles, not matrices.
\end{itemize}

\chapter{Documentation}\label{documentation}

\begin{itemize}
\item
  vignette should include not only the coefficients as output in an
  example, but also those coefficients written up as a general latex
  model and as a latex model with those specific coefficients
  substituted in
\item
  \textbf{show} your example data in the README so users immediately see
  the structure
\end{itemize}

function to write out model form and fitted model in latex for sanity
checking: some sort of model\_report / model\_form generic

it's a bad idea to expect users to learn the \emph{math} for your model
from function level documentation, or math presented in ascii or unicode
or poorly rendered latex.

show write out the math in a nicely formatted vignette, and then clearly
describe the connection between code objects and math objects there as
well

\chapter{Functional programming
principles}\label{functional-programming-principles}

calls to fit should be pure: i.e.~no side effects like plotting, and
especially no plotting with invisible object return - side effects:
useful in interactive mode, irritating in programmatic mode

\begin{itemize}
\tightlist
\item
  type safety, particularly of returned objects
\item
  type safety with respect to single fits vs sets of fits
\end{itemize}

\chapter{Implementation}\label{implementation}

\begin{itemize}
\tightlist
\item
  argument matching for categorical choices
\end{itemize}

\chapter{Interactive modelling}\label{interactive-modelling}

what most people do different because there's a person looking at stuff
as opposed to programmatic model when it's just code interacting with
the model with no human involved

this is a chapter mostly to remind us to think of differences between
the two and how they might be important in terms of interface

\chapter{Interface}\label{interface}

\begin{itemize}
\tightlist
\item
  user friendly interfaces
\end{itemize}

good and bad existing idioms

\begin{itemize}
\tightlist
\item
  methods to implement
\item
  examples of tried and true workflows
\end{itemize}

methods to implement - note on plotting: Should be easy to get the
values plotted so others can make their own plots

TWO DISTINCT ISSUES THAT GET RESOLVED IN FORMULAE:

\begin{verbatim}
design matrix specification
model specification. (a la fGarch::garchFit(~arma(1, 1) + garch(1, 0)))
\end{verbatim}

\chapter{Low and high level
interfaces}\label{low-and-high-level-interfaces}

\begin{itemize}
\tightlist
\item
  high level versus low level interface
\item
  programmatic versus interactive use
\end{itemize}

when you should use which

examples: - high level: keras, brms - low level: tensorflow, stan

\chapter{Model objects}\label{model-objects}

\begin{itemize}
\item
  some explanation of why and how to save the function call
\item
  generally what kinds of things should go into a model object, giving
  model objects a class so other people can extend them
\item
  S3 object creation and validation for model building a la Advanced R
\item
  Model classes beyond lists. when is S4 worth it? when is R6?
\item
  Every modeling function should include its package version in its data
  object I will now save my models as a list of three objects: model,
  data, and sessioninfo::session\_info()
\end{itemize}

\chapter{Programmatic modelling}\label{programmatic-modelling}

i.e.~interacting with models programmatically

examples: - packages that export a model from someone to use a la
\href{https://github.com/mkearney/tweetbotornot}{\texttt{botornot}} -
models sitting behind a Plumber API - etc

\chapter{References}\label{references}

\begin{itemize}
\tightlist
\item
  bdr's
  \href{https://developer.r-project.org/model-fitting-functions.html}{model
  fitting functions in r}
\end{itemize}

\chapter{Testing}\label{testing}

\begin{itemize}
\tightlist
\item
  testing against existing software - say a Matlab implementation
\item
  saving long running models in \texttt{R/sysdata.rda} with
  \texttt{usethis::use\_data(model\_obj,\ internal\ =\ TRUE)}
\end{itemize}

\chapter{Workflow}\label{workflow}

\section{Prediction}\label{prediction}

\begin{enumerate}
\def\labelenumi{\arabic{enumi}.}
\tightlist
\item
  feature engineering
\item
  ML wizardry
\item
  more feature engineering
\item
  ???
\item
  predictions
\end{enumerate}

\section{Inference}\label{inference}

\begin{enumerate}
\def\labelenumi{\arabic{enumi}.}
\tightlist
\item
  Clean data
\item
  Specify model
\item
  Fit model
\item
  Check that model fitting process converged / worked
\item
  Check statistical assumptions of model
\end{enumerate}

KEY part that always gets left out: working with multiple modellings

\bibliography{book.bib,packages.bib}


\end{document}
